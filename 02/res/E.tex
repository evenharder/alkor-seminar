\documentclass[hyperref={unicode}]{beamer}

\usetheme{CambridgeUS}
\usecolortheme{beaver}
\usepackage[hangul]{kotex}

\usepackage{tabularx}
\usepackage{xcolor}
\usepackage{graphicx}
\usepackage[normalem]{ulem}
\usepackage{amsmath, amssymb}
\usepackage{colortbl}
\usepackage{textcomp}
\usepackage{multirow}
\setbeamertemplate{items}[circle]
\usepackage{minted}


\begin{document}
 
    \begin{frame}
        \onslide<1-> 2-친구와 3-친구 관계를 모두 없애야 한다.
        \begin{itemize}
        \item<2-> 2-친구의 경우, A랑도 B랑도 친구인 경우이므로 없애야 한다.
        \end{itemize}
        \onslide<3-> 문제가 되는 것은 3-친구 관계.
        \begin{itemize}
            \item<4-> A의 친구들과 B의 친구들끼리 연결된 그래프에서, 최소한의 친구들(과 각자의 연결관계)을 제거해서 선분을 모두 없애야 한다.
        \end{itemize}
    \end{frame}
    
    \begin{frame}
        사실 대단히 전형적인 문제...
        \begin{itemize}
            \item<1-> 3-친구 제거 문제는 이분 그래프에서의 vertex cover 문제와 동치이다.
            \item<2-> 그리고 이분 그래프에서 minimum vertex cover = maximum matching (K\"onig's Theorem)
            \item<3-> 즉 이분 그래프에서의 매칭을 진행하면 된다.
        \end{itemize}
        \onslide<4-> 네트워크 플로우 이론을 이용한 간단한 $ O(VE) = O(N^{\,3}) $ 이분 매칭 알고리즘이 존재한다.
        
    \end{frame}

    \begin{frame}
        Code : \url{http://boj.kr/4a3936b589a8442faf55800884e9759a}
        \begin{itemize}
            \item {\color{blue} \href{https://kks227.blog.me/220804885235}{네트워크 플로우 관련 블로그 (kks227)}}
            \item {\color{blue}\href{https://kks227.blog.me/220807541506}{이분 매칭 관련 블로그 (kks227)}}
            \item {\color{blue} \href{https://kks227.blog.me/220816033373}{Hopcroft-Karp Algorithm (kks227)}} : 시간복잡도 $ O(EV^{\,0.5}) $.
        \end{itemize}
    \end{frame}
\end{document}