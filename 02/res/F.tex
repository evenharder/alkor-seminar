\documentclass[hyperref={unicode}]{beamer}

\usetheme{CambridgeUS}
\usecolortheme{beaver}
\usepackage[hangul]{kotex}

\usepackage{tabularx}
\usepackage{xcolor}
\usepackage{graphicx}
\usepackage[normalem]{ulem}
\usepackage{amsmath, amssymb}
\usepackage{colortbl}
\usepackage{textcomp}
\usepackage{multirow}
\setbeamertemplate{items}[circle]
\usepackage{minted}


\begin{document}
 
    \begin{frame}
        \onslide<1-> 직선이 평면에 있는 점들을 분할하는지를 확인하자.
        \begin{itemize}
            \item<2-> convex hull을 만들고, 직선이 이를 통과하는지 판별하는 것과 동치.
        \end{itemize}
        \onslide<3-> standard한 문제이기에 다양한 풀이가 있다.
        \begin{itemize}
            \item<4-> 여기서는 보다 간결한 이분탐색을 이용한 풀이를 설명한다.
        \end{itemize}
        
    \end{frame}
    
    \begin{frame}
        \onslide<1-> convex hull을 통과하는 방향을 생각해보자.
        \begin{itemize}
            \item<2-> 이 방향에서 가장 멀리 떨어진 두 점을 생각해볼 수 있다.
            \item<3-> 각 방향별로 이 두 점을 알면...?
            \begin{itemize}
                \item<4-> 두 점을 연결한 선분과 직선이 교차하는지 판별하면 된다!
            \end{itemize}
            \item<5-> 그럼 각도를 조금씩 기울여보자. 언제 두 점이 바뀔까?
            \item<6-> 직선의 기울기가 두 점이 속한 convex hull의 선분의 기울기를 넘어서면 된다!
        \end{itemize}
    \end{frame}
    
    \begin{frame}
        \begin{center}
            \begin{tikzpicture}[line cap=round,line join=round,>=triangle 45,x=1cm,y=1cm, scale=0.3]
            \clip(-18.896892689405195,-12.518551037588734) rectangle (0.6303288827343082,10.016027443340084);
            \draw [line width=1.5pt] (-11.22606159565516,4.5915000519107005)-- (-6.890421054925437,4.006919304846019);
            \draw [line width=1.5pt] (-6.890421054925437,4.006919304846019)-- (-5.0960082449606094,1.6531034367065363);
            \draw [line width=1.5pt] (-5.0960082449606094,1.6531034367065363)-- (-4.18,-2.5);
            \draw [line width=1.5pt] (-4.18,-2.5)-- (-6.372730499264858,-7.09482217668112);
            \draw [line width=1.5pt] (-6.372730499264858,-7.09482217668112)-- (-10.616362022738443,-8.852047409064667);
            \draw [line width=1.5pt] (-10.616362022738443,-8.852047409064667)-- (-12.8,-7.46);
            \draw [line width=1.5pt] (-12.8,-7.46)-- (-13.94,-3.56);
            \draw [line width=1.5pt] (-11.22606159565516,4.5915000519107005)-- (-13.94,-3.56);
            \draw [line width=1.5pt,color=ccwwff,domain=-18.896892689405195:0.6303288827343082] plot(\x,{(-2.2823173875755023--1.235346774086635*\x)/8.09154325096045});
            \begin{scriptsize}
            \draw [fill=ffffff] (-11.22606159565516,4.5915000519107005) circle (10pt);
            \draw [fill=wqwqwq] (-6.890421054925437,4.006919304846019) circle (10pt);
            \draw [fill=wqwqwq] (-5.0960082449606094,1.6531034367065363) circle (10pt);
            \draw [fill=wqwqwq] (-4.18,-2.5) circle (10pt);
            \draw [fill=wqwqwq] (-6.372730499264858,-7.09482217668112) circle (10pt);
            \draw [fill=ffffff] (-10.616362022738443,-8.852047409064667) circle (10pt);
            \draw [fill=wqwqwq] (-12.8,-7.46) circle (10pt);
            \draw [fill=wqwqwq] (-13.94,-3.56) circle (10pt);
            \end{scriptsize}
            \end{tikzpicture}
        \end{center}
    \end{frame}
    
    \begin{frame}
        \begin{center}
            \begin{tikzpicture}[line cap=round,line join=round,>=triangle 45,x=1cm,y=1cm, scale=0.3]
            \clip(-18.896892689405195,-12.51855103758873) rectangle (0.6303288827343082,10.016027443340082);
            \draw [line width=1.5pt] (-11.22606159565516,4.5915000519107005)-- (-6.890421054925437,4.006919304846019);
            \draw [line width=1.5pt] (-6.890421054925437,4.006919304846019)-- (-5.0960082449606094,1.6531034367065363);
            \draw [line width=1.5pt] (-5.0960082449606094,1.6531034367065363)-- (-4.18,-2.5);
            \draw [line width=1.5pt] (-4.18,-2.5)-- (-6.372730499264858,-7.09482217668112);
            \draw [line width=1.5pt] (-6.372730499264858,-7.09482217668112)-- (-10.616362022738443,-8.852047409064667);
            \draw [line width=1.5pt] (-10.616362022738443,-8.852047409064667)-- (-12.8,-7.46);
            \draw [line width=1.5pt] (-12.8,-7.46)-- (-13.94,-3.56);
            \draw [line width=1.5pt] (-11.22606159565516,4.5915000519107005)-- (-13.94,-3.56);
            \draw [line width=1.5pt,dotted,color=ccwwff,domain=-18.896892689405195:0.6303288827343082] plot(\x,{(-2.2701--1.24*\x)/8.09});
            \draw [line width=1.5pt,color=ccwwff,domain=-18.896892689405195:0.6303288827343082] plot(\x,{(--30.476411575757446--4.170149386117076*\x)/7.042411099720561});
            \begin{scriptsize}
            \draw [fill=ffffff] (-11.22606159565516,4.5915000519107005) circle (10pt);
            \draw [fill=wqwqwq] (-6.890421054925437,4.006919304846019) circle (10pt);
            \draw [fill=wqwqwq] (-5.0960082449606094,1.6531034367065363) circle (10pt);
            \draw [fill=wqwqwq] (-4.18,-2.5) circle (10pt);
            \draw [fill=ffffff] (-6.372730499264858,-7.09482217668112) circle (10pt);
            \draw [fill=wqwqwq] (-10.616362022738443,-8.852047409064667) circle (10pt);
            \draw [fill=wqwqwq] (-12.8,-7.46) circle (10pt);
            \draw [fill=wqwqwq] (-13.94,-3.56) circle (10pt);
            \end{scriptsize}
            \end{tikzpicture}
        \end{center}
    \end{frame}
    
    \begin{frame}
        \onslide<1-> convex hull을 위쪽과 아래쪽으로 쪼개자.
        \begin{itemize}
            \item<2-> Andrew's Monotone Chain Algorithm으로 쉽게 구축 가능.
        \end{itemize}
        \onslide<3-> 양 반껍질은 기울기 순으로 정렬이 되어 있다.
        \begin{itemize}
        \item<4-> 여기서 이분탐색을 진행해서 해당 기울기 이상이 되는 점을 잡자.
        \item<5-> 이 점이 주어진 직선의 기울기로부터 `가장 멀리 떨어져있는 점'이 된다! 
        \item<6-> 선분과 직선이 만나는지 확인만 하면 끝.
        \end{itemize}
    \end{frame}
    
    \begin{frame}
        Code : \url{http://boj.kr/514a4fadbe0d4521b22c45bef6bb8cea}
        \begin{itemize}
            \item {\color{blue}\href{https://en.wikibooks.org/wiki/Algorithm_Implementation/Geometry/Convex_hull/Monotone_chain}{Andrew's Monotone Chain Algorithm}}
            \item CCW도 익숙해집시다
        \end{itemize}
        
    \end{frame}
\end{document}