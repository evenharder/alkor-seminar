\documentclass[hyperref={unicode}]{beamer}

\usetheme{CambridgeUS}
\usecolortheme{beaver}
\usepackage[hangul]{kotex}

\usepackage{tabularx}
\usepackage{xcolor}
\usepackage{graphicx}
\usepackage[normalem]{ulem}
\usepackage{amsmath, amssymb}
\usepackage{colortbl}
\usepackage{textcomp}
\usepackage{multirow}
\setbeamertemplate{items}[circle]
\usepackage{minted}


\begin{document}
 
    \begin{frame}[fragile]
        \onslide<1-> FYI : bitwise xor(exclusive or)는 보통 \verb|^|(caret)을 씁니다.
    
        \onslide<2-> $ X_1 \oplus X_2 \oplus \cdots \oplus X_k = Y_1 \oplus Y_2 \oplus \cdots \oplus Y_{n-k}$?
        \begin{itemize}
        \item<3-> $ a_1 \oplus a_2 \oplus \cdots \oplus a_n = 0 $와 동치.
        \end{itemize}
        \onslide<4-> 즉, 모든 수를 xor해서 0일 때만 위의 조건을 만족하며 분할할 수 있다.
        \begin{itemize}
        \item<5-> 분할이 가능하면 집합 $ Y $에는 최소값만 넣으면 된다.
        \end{itemize}
        \onslide<6> 문제 이름 그대로 속이는, 좋은 낚시 문제입니다. 항상 지문을 꼼꼼히 읽읍시다!
    \end{frame}
    
    \begin{frame}
        Code : \url{http://boj.kr/f9a9fb1dcd1a4f3bbbaf6f1ac8960e51}
    \end{frame}

\end{document}