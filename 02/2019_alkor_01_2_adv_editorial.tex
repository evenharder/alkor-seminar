%!TeX TXS-program:compile = txs:///xelatex/[--shell-escape]
\documentclass[hyperref={unicode}]{beamer}

\usetheme{CambridgeUS}
\usecolortheme{beaver}
\usepackage[hangul]{kotex}
\usepackage{standalone}
\usepackage{tabularx}
\usepackage{xcolor}
\usepackage{graphicx}
\usepackage[normalem]{ulem}
\usepackage{amsmath, amssymb}
\usepackage{colortbl}
\usepackage{textcomp}
\usepackage{multirow}

\usepackage{pgf,tikz,pgfplots}
\pgfplotsset{compat=1.15}
\usepackage{mathrsfs}
\usetikzlibrary{arrows}
\definecolor{ccwwff}{rgb}{0.8,0.4,1}
\definecolor{wqwqwq}{rgb}{0.3764705882352941,0.3764705882352941,0.3764705882352941}
\definecolor{ffffff}{rgb}{1,1,1}

\setbeamertemplate{items}[circle] % 
\usepackage{minted}
\setminted{breaklines=true, tabsize = 2, breaksymbolleft=}
\usemintedstyle{perldoc}

\newcolumntype{Y}{>{\centering\arraybackslash}X}
\usefonttheme[stillsansserifmath]{serif}
\setmainfont{Noto Sans CJK KR}
\setmonofont{Ubuntu Mono}

\definecolor{crimsonred}{HTML}{760023}
\definecolor{akw}{HTML}{009f6b}
\definecolor{xmf}{HTML}{e74c3c}

\newcommand{\insertsectionpage}[4]{
  \begin{frame}
  %\vfill
  \centering
  \begin{beamercolorbox}[sep=8pt,center,rounded=true]{title}
    \usebeamerfont{title}\textbf{\insertsectionhead}\par%
    \insertsubsectionhead%
  \end{beamercolorbox}
  \ifx&#1&%
  \else%
  {\small출처 : #1}\par\vskip 5pt%
  \fi%
  {\scriptsize제출 : #2 / \scriptsize정답 : {\color{akw}#3}}\par%
  {\scriptsize평균 시도 : #4}\par%
  {\scriptsize}
  \vfill
  \end{frame}
}
\setbeamertemplate{navigation symbols}{}

%
\makeatletter
\def\th@lemstyle{%
    \normalfont % body font
    \setbeamercolor{block title example}{bg=orange,fg=white}
    \setbeamercolor{block body example}{bg=orange!20,fg=black}
    \def\inserttheoremblockenv{exampleblock}
  }
\makeatother
\newcommand{\minigray}{\scriptsize\color{gray}}
\newcommand{\smallgray}{\footnotesize\color{gray}}
\theoremstyle{lemstyle}
\newtheorem*{mylemma}{보조정리}

\graphicspath{{./res/}}
\title[1학기 고급반 2주차 모의고사 풀이]{2019 AlKor 1학기 고급반 2주차 모의고사 풀이}
\subtitle{고려대학교 정보보호학부 알고리즘 문제해결 동아리}
\author[2019 AlKor]{Prepared by \texttt{@evenharder}}
\date{2019년 4월 6일}
%\institute{2019 AlKor}
\begin{document}
    
    \begin{frame}
        \titlepage
    \end{frame}
    
    \begin{frame}{머릿말}
        \begin{itemize}
            \item \textbf{풀이를 보기 전에 문제에 대해 더 생각을 해보세요!}
            \item {\bf \color{blue}\href{https://www.acmicpc.net/blog/view/55}{BOJ 101}}과 {\bf \color{blue}\href{https://www.acmicpc.net/blog/view/70}{자주 틀리는 요인}} 문서는 숙지하셔야 합니다.
            \item 곰곰히 생각해보면 풀 수 있는 문제들로 구성했습니다.
            \item 1주차보다 쉽게 했고, 보다 만족스러운 참여도와 결과가 보였습니다. 처음 3문제는 모두 접근해볼만한 난이도로 준비하는 게 좋아보입니다.
            \begin{itemize}
            \item 풀이 순서는 제가 느낀 난이도순이지만 애매한 부분이 많습니다.
            \end{itemize}
        \end{itemize}
    \end{frame}
    
    \begin{frame}{머릿말}
        \begin{itemize}
            \item 모든 문제를 지금 푸는 건 어려울 수 있습니다. 하나하나씩 풀어보세요.
            \item 때문에 (가능하면) 각 문제 별로 정답 코드를 제공하려고 합니다.
            \begin{itemize}
            \item 대부분 정해 코드가 공식 대회 사이트에 있기에 비공개할 이유가 없습니다.
            \item 코드를 보면 풀이를 보다 쉽게 이해할 수 있습니다.
            \item 물론 어느 상황에서든 \textbf{표절은 절대 금지입니다.}
            \end{itemize}
            \item 제출, 정답 및 정답율은 2019년 4월 6일 16:00 (UTC +9) 기준입니다.
            \item {\smallgray 힘들어서 해설지 퀄리티가 점점 하락하고 있습니다...}
        \end{itemize}
    \end{frame}
    %%%%%%%%%%%%%%%%%%%%%%%%%%%%%%%%%%%%%%%%%%%%%%%%%%%%%%%%%%%%%%%%%%%%%
    %
    \section{Problem A}
    \subsection{속이기 (BOJ 11895)}
    \insertsectionpage{(oj.uz) GA5 1번}{15}{15}{1.27}
    \documentclass[hyperref={unicode}]{beamer}

\usetheme{CambridgeUS}
\usecolortheme{beaver}
\usepackage[hangul]{kotex}

\usepackage{tabularx}
\usepackage{xcolor}
\usepackage{graphicx}
\usepackage[normalem]{ulem}
\usepackage{amsmath, amssymb}
\usepackage{colortbl}
\usepackage{textcomp}
\usepackage{multirow}
\setbeamertemplate{items}[circle]
\usepackage{minted}


\begin{document}
 
    \begin{frame}[fragile]
        \onslide<1-> FYI : bitwise xor(exclusive or)는 보통 \verb|^|(caret)을 씁니다.
    
        \onslide<2-> $ X_1 \oplus X_2 \oplus \cdots \oplus X_k = Y_1 \oplus Y_2 \oplus \cdots \oplus Y_{n-k}$?
        \begin{itemize}
        \item<3-> $ a_1 \oplus a_2 \oplus \cdots \oplus a_n = 0 $와 동치.
        \end{itemize}
        \onslide<4-> 즉, 모든 수를 xor해서 0일 때만 위의 조건을 만족하며 분할할 수 있다.
        \begin{itemize}
        \item<5-> 분할이 가능하면 집합 $ Y $에는 최소값만 넣으면 된다.
        \end{itemize}
        \onslide<6> 문제 이름 그대로 속이는, 좋은 낚시 문제입니다. 항상 지문을 꼼꼼히 읽읍시다!
    \end{frame}
    
    \begin{frame}
        Code : \url{http://boj.kr/f9a9fb1dcd1a4f3bbbaf6f1ac8960e51}
    \end{frame}

\end{document}
    
    %%%%%%%%%%%%%%%%%%%%%%%%%%%%%%%%%%%%%%%%%%%%%%%%%%%%%%%%%%%%%%%%%%%%%%%%%%%
    
    \section{Problem C}
    \subsection{이진 트리 (BOJ 13325)} 
    \insertsectionpage{ACM ICPC Daejeon Nationalwide Internet Competition 2016 A번}{10}{10}{2}
    \documentclass[hyperref={unicode}]{beamer}

\usetheme{CambridgeUS}
\usecolortheme{beaver}
\usepackage[hangul]{kotex}

\usepackage{tabularx}
\usepackage{xcolor}
\usepackage{graphicx}
\usepackage[normalem]{ulem}
\usepackage{amsmath, amssymb}
\usepackage{colortbl}
\usepackage{textcomp}
\usepackage{multirow}
\setbeamertemplate{items}[circle]
\usepackage{minted}


\begin{document}
 
    \begin{frame}
        \onslide<1-> 포화이진트리(perfect binary tree)가 주어진다.
        \begin{itemize}
            \item<2-> root를 1번째로 두면, $ x $번째 정점의 자식은 $ 2x $번째, $ 2x+1 $번째
        \end{itemize}
        \onslide<3-> leaf 바로 위의 정점들부터 보자.
        \begin{itemize}
            \item<4-> 정점 $ x $에 대해, 한쪽은 길이가 $ d[2x] $, 한쪽은 $ d[2x+1] $
            \item<5-> 이 경우 적은 쪽에 차만큼 더해주어야 한다.
            \item<6-> 그러면 그쪽 경로는 $ \max(d[2x], d[2x+1]) $로 통일된다.
            \item<7-> 이를 계속 반복하면 한 level를 해결 가능하고, 그럼 바로 위의 level도 마찬가지로...!
        \end{itemize}
        
    \end{frame}
    
    \begin{frame}
        \url{http://boj.kr/6535d79448104e9cb7e2e7ca225ec804}
                  
        \onslide  
        \begin{itemize}
            \item $ 2^n - 1 $에서 $ 1 $까지 줄여가면 모든 정점을 순회 가능하다!
            \item 종종 쓰이는 트릭이니 (예시 : segment tree) 잘 알아두자.
        \end{itemize}
    \end{frame}
\end{document}    

    %%%%%%%%%%%%%%%%%%%%%%%%%%%%%%%%%%%%%%%%%%%%%%%%%%%%%%%%%%%%%%%%%%%%%%%%%%%    
    \section{Problem B}
    \subsection{순열 그래프의 연결성 판별 (BOJ 7982)}
    \insertsectionpage{AMPPZ 2012 I번}{13}{10}{3.6}
    \documentclass[hyperref={unicode}]{beamer}

\usetheme{CambridgeUS}
\usecolortheme{beaver}
\usepackage[hangul]{kotex}

\usepackage{tabularx}
\usepackage{xcolor}
\usepackage{graphicx}
\usepackage[normalem]{ulem}
\usepackage{amsmath, amssymb}
\usepackage{colortbl}
\usepackage{textcomp}
\usepackage{multirow}
\setbeamertemplate{items}[circle]
\usepackage{minted}


\begin{document}
 
    \begin{frame}
        \onslide<1-> 그래프는 수열 $ \{a_i\}  $에서 연결해서 생성할 수 있다. 뭘 알 수 있을까?
        \begin{itemize}
        \item<2-> 컴포넌트는 \textbf{연속}되어 나타난다.
        \item<3-> 어떤 구간 $ L $이 컴포넌트가 되었다고 가정하자. 그러면
            \begin{itemize}
            \item<4-> $ L $ 뒤로는 $ L $의 최대 원소보다 큰 값만 와야 하며
            \item<5-> $ L $ 앞에는 $ L $의 최소 원소보다 작은 값만 와야 한다.
            \end{itemize}
        \item<6-> 조금 정리해보면, $ \max(a_1, a_2, \cdots, a_k) = k$ 일 때 컴포넌트가 분리된다.
        \begin{itemize}
        \item<7-> $ \max(a_1, a_2, \cdots, a_k) = k$이면 $ \{a_1, a_2, \cdots, a_k\} = \{1, 2, \cdots, k\}$
        \end{itemize}
        \end{itemize}
        \onslide<7-> 정답 출력은 각 컴포넌트의 최대 원소를 저장하면 어렵지 않게 할 수 있다.
        
    \end{frame}  
    

    \begin{frame}
        Code : \url{http://boj.kr/f88c861f7c114fabbb214e24840d2fd5}
    \end{frame}
\end{document}
    
    %%%%%%%%%%%%%%%%%%%%%%%%%%%%%%%%%%%%%%%%%%%%%%%%%%%%%%%%%%%%%%%%%%%%%%%%%%%
    
    \section{Problem D}
    \subsection{Bricks (BOJ 10510)}
    \insertsectionpage{CERC 2014 I번}{9}{8}{3.38}
    \documentclass[hyperref={unicode}]{beamer}

\usetheme{CambridgeUS}
\usecolortheme{beaver}
\usepackage[hangul]{kotex}

\usepackage{tabularx}
\usepackage{xcolor}
\usepackage{graphicx}
\usepackage[normalem]{ulem}
\usepackage{amsmath, amssymb}
\usepackage{colortbl}
\usepackage{textcomp}
\usepackage{multirow}
\setbeamertemplate{items}[circle]
\usepackage{minted}


\begin{document}
 
    \begin{frame}
        \onslide<1-> 검은 벽돌이 $ b $개, 흰 벽돌이 $ w $개 있으면 나눌 수 있는 비율은 $ \frac{b}{\gcd(b, w)} : \frac{w}{\gcd(b, w)} $.
        \begin{itemize}
        \item<2> 해당 비율을 만족하며 벽돌을 분리할 수 있으면 분리해야 한다.
        \end{itemize}
    \end{frame}
    
    \begin{frame}
        \onslide<1-> 기하학적으로 좌표평면에 놓고 접근해보자.
        \begin{itemize}
            \item<2-> 좌표평면의 원점에서 시작해서 \texttt{B} 1개당 $ x $좌표, \texttt{W} 1개당 $ y $좌표 1 증가
            \item<3-> 이 때 $ wx - by = 0 $과의 교점 중 격자점의 개수를 구하면 된다.
        \end{itemize}
        \onslide<4-> 선분 하나하나씩 옮겨가며 따지면 $ O(N) $에 해결 가능.
        
        \onslide<5-> 문제를 굳이 이렇게 바꾸지 않아도 구현은 비슷할 것 같지만 예외처리가 곤혹스러울 수 있다.
    \end{frame}

    \begin{frame}
        Code : \url{http://boj.kr/742141a744d1427cb61e2e782ed0f74c}
    \end{frame}
\end{document}
    
    %%%%%%%%%%%%%%%%%%%%%%%%%%%%%%%%%%%%%%%%%%%%%%%%%%%%%%%%%%%%%%%%%%%%%%%%%%%
    
    \section{Problem H}
    \subsection{생물학자 (BOJ 3116)}
    \insertsectionpage{Croatian Highschool Competitions in Informatics 2009 National Competition \#2 Seniors 1번}{7}{7}{2.29}
    \documentclass[hyperref={unicode}]{beamer}

\usetheme{CambridgeUS}
\usecolortheme{beaver}
\usepackage[hangul]{kotex}

\usepackage{tabularx}
\usepackage{xcolor}
\usepackage{graphicx}
\usepackage[normalem]{ulem}
\usepackage{amsmath, amssymb}
\usepackage{colortbl}
\usepackage{textcomp}
\usepackage{multirow}
\setbeamertemplate{items}[circle]
\usepackage{minted}


\begin{document}
    
    \begin{frame}
        \onslide<1-> 서로 다른 두 박테리아의 쌍에 대해
        \begin{itemize}
            \item<2-> 이 두 박테리아가 만나는지
            \item<3-> 만날 경우, 그 자리에 있는 다른 박테리아가 있는지
        \end{itemize}
        \onslide<3-> 를 계산해볼 수 있다.
    \end{frame}
    
    \begin{frame}
        \onslide<1-> 두 박테리아가 만나는지는 어떻게 확인할 수 있을까?
        \begin{itemize}
            \item<2-> 상대 위치와 상대속도를 고려하면 확인 가능.
        \end{itemize}
        \onslide<3-> 그럼 이 때 다른 박테리아가 있는지는?
        \begin{itemize}
            \item<4-> 만날 수 있는 박테리아의 초기 위치와 방향은 {\smallgray(각 방향별로)} 8가지밖에 없다!
            \item<5-> 이분 탐색이나 mapping 등을 통해 확인 가능.
        \end{itemize}
        \onslide<6-> 박테리아 쌍은 $ O(N^{\,2}) $개 있으므로 총 시간복잡도 $ O(N^{\,2} \lg N) $에 가능하다.
    \end{frame}
    
    \begin{frame}
        Code : \url{http://boj.kr/c9fdc71d9b244ec3a20b60acbe4841d4}
    \end{frame}
    
\end{document}
    
    %%%%%%%%%%%%%%%%%%%%%%%%%%%%%%%%%%%%%%%%%%%%%%%%%%%%%%%%%%%%%%%%%%%%%%%%%%%
    
    \section{Problem G}
    \subsection{로봇 조종하기 (BOJ 2169)}
    \insertsectionpage{KOI 2002 고등부 1번}{9}{9}{2.11}
    \documentclass[hyperref={unicode}]{beamer}

\usetheme{CambridgeUS}
\usecolortheme{beaver}
\usepackage[hangul]{kotex}

\usepackage{tabularx}
\usepackage{xcolor}
\usepackage{graphicx}
\usepackage[normalem]{ulem}
\usepackage{amsmath, amssymb}
\usepackage{colortbl}
\usepackage{textcomp}
\usepackage{multirow}
\setbeamertemplate{items}[circle]
\usepackage{minted}


\begin{document}
 
    \begin{frame}
        \onslide<1-> 기본적으로 `칸 $ (i, j) $까지 거치면서 얻을 수 있는 최대 가치'을 저장하는 동적 계획법으로 접근해야 한다.
        \begin{itemize}
            \item<2-> 칸에 도달하는 방법 : 왼쪽, 오른쪽, (위).
            \item<3-> 현재 칸 기준으로 왼쪽 또는 바로 위에서 오는 경우만 고려해보자.
            \begin{itemize}
                \item<4-> 그럼 $ l[i][j] = min(l[i][j-1], dp[i-1][j]) + a[i][j] $.
                \item<5-> $ (i, j-1) $ 또는 $ (i-1, j) $를 방문해야 하기 때문!
                \item<5-> 오른쪽에서 오는 경우는 고려할 필요가 없다 {\minigray(한 번 밟은 칸은 다시 밟을 수 없음)}
            \end{itemize}
            \item<6-> 오른쪽도 비슷하게 정의할 수 있다.
            \item<7-> 각 칸별로 두 값의 최대값을 취하면 된다. 행마다 반복하자.
        \end{itemize}
        \onslide<8-> 총 시간 복잡도는 $ O(NM) $.
    \end{frame}
    
    \begin{frame}
        Code : \url{http://boj.kr/729bb42602d14ee88b7e81568b9dce9b}
        \begin{itemize}
            \item 토글링(toggling)을 이용한 공간 복잡도 $ O(M) $ 코드
            \item 이렇게 바로 전 열의 상태만 알아도 되는 경우 공간 복잡도(와 약간의 수행시간)을 줄일 수 있습니다.
        \end{itemize}
    \end{frame}


\end{document}
    
    %%%%%%%%%%%%%%%%%%%%%%%%%%%%%%%%%%%%%%%%%%%%%%%%%%%%%%%%%%%%%%%%%%%%%%%%%%%
    
    \section{Problem E}
    \subsection{알 수도 있는 사람 (BOJ 13358)}
    \insertsectionpage{???}{5}{5}{2.4}
    \documentclass[hyperref={unicode}]{beamer}

\usetheme{CambridgeUS}
\usecolortheme{beaver}
\usepackage[hangul]{kotex}

\usepackage{tabularx}
\usepackage{xcolor}
\usepackage{graphicx}
\usepackage[normalem]{ulem}
\usepackage{amsmath, amssymb}
\usepackage{colortbl}
\usepackage{textcomp}
\usepackage{multirow}
\setbeamertemplate{items}[circle]
\usepackage{minted}


\begin{document}
 
    \begin{frame}
        \onslide<1-> 2-친구와 3-친구 관계를 모두 없애야 한다.
        \begin{itemize}
        \item<2-> 2-친구의 경우, A랑도 B랑도 친구인 경우이므로 없애야 한다.
        \end{itemize}
        \onslide<3-> 문제가 되는 것은 3-친구 관계.
        \begin{itemize}
            \item<4-> A의 친구들과 B의 친구들끼리 연결된 그래프에서, 최소한의 친구들(과 각자의 연결관계)을 제거해서 선분을 모두 없애야 한다.
        \end{itemize}
    \end{frame}
    
    \begin{frame}
        사실 대단히 전형적인 문제...
        \begin{itemize}
            \item<1-> 3-친구 제거 문제는 이분 그래프에서의 vertex cover 문제와 동치이다.
            \item<2-> 그리고 이분 그래프에서 minimum vertex cover = maximum matching (K\"onig's Theorem)
            \item<3-> 즉 이분 그래프에서의 매칭을 진행하면 된다.
        \end{itemize}
        \onslide<4-> 네트워크 플로우 이론을 이용한 간단한 $ O(VE) = O(N^{\,3}) $ 이분 매칭 알고리즘이 존재한다.
        
    \end{frame}

    \begin{frame}
        Code : \url{http://boj.kr/4a3936b589a8442faf55800884e9759a}
        \begin{itemize}
            \item {\color{blue} \href{https://kks227.blog.me/220804885235}{네트워크 플로우 관련 블로그 (kks227)}}
            \item {\color{blue}\href{https://kks227.blog.me/220807541506}{이분 매칭 관련 블로그 (kks227)}}
            \item {\color{blue} \href{https://kks227.blog.me/220816033373}{Hopcroft-Karp Algorithm (kks227)}} : 시간복잡도 $ O(EV^{\,0.5}) $.
        \end{itemize}
    \end{frame}
\end{document}
    
    %%%%%%%%%%%%%%%%%%%%%%%%%%%%%%%%%%%%%%%%%%%%%%%%%%%%%%%%%%%%%%%%%%%%%%%%%%%
    
    \section{Problem I}
    \subsection{장난감 정리 로봇 (BOJ 8875)}
    \insertsectionpage{IOI 2013 5번}{4}{4}{2.5}
    \documentclass[hyperref={unicode}]{beamer}

\usetheme{CambridgeUS}
\usecolortheme{beaver}
\usepackage[hangul]{kotex}

\usepackage{tabularx}
\usepackage{xcolor}
\usepackage{graphicx}
\usepackage[normalem]{ulem}
\usepackage{amsmath, amssymb}
\usepackage{colortbl}
\usepackage{textcomp}
\usepackage{multirow}
\setbeamertemplate{items}[circle]
\usepackage{minted}


\begin{document}
 
    \begin{frame}
        \onslide<1-> 모든 장난감 로봇이 각자 $ k $개 이하의 장난감을 정리하면서 모든 장난감을 치울 수 있는지 판별할 수 있을까?
        \begin{itemize}
            \item<2-> Yes! Parametric search를 해보자
        \end{itemize}
        \onslide<3-> 기본적인 접근은 그리디.
        \begin{itemize}
            \item<4-> 연약한 로봇은 크기에 상관없이, 작은 로봇은 무게에 상관없이 짐을 들 수 있다. 
        \end{itemize}
    \end{frame}
    
    \begin{frame}
        \onslide<1-> 우선 연약한(무게 제한이 있는) 로봇부터 생각해보자.
        \begin{itemize}
            \item<2-> $ x < y$일 때, 제한 $ x $인 연약한 로봇이 정리할 수 있는 장난감은 제한 $ y $인 연약한 로봇도 정리할 수 있다.
            \item<3-> 그러므로 (무게) 제한에 대한 오름차순으로 연약한 로봇을 정렬하자.
            \item<4-> 그럼 각 로봇이 추가되면서 제거할 수 있는 장난감들이 (무게에 따라) 추가된다.
            \item<5-> 이 상태에서, 각 로봇당 가장 큰 장난감 $ k $개를 제거하자.
        \end{itemize}
        
    \end{frame}
    
    \begin{frame}
        \onslide<1-> 왜 이렇게 할까?
            \begin{itemize}
            \item<2-> 나머지를 맡을 작은 로봇들은 무게의 제약을 받지 않기 때문에 큰 장난감부터 정리해야 한다. 
            \end{itemize}
        \onslide<3-> 이젠 작은(크기 제한이 있는) 로봇을 생각해볼 차례.
        \begin{itemize}
            \item<4-> 제한이 큰 로봇부터 가장 무거운 장난감들을 최대 $ k $개까지 정리한다.
            \item<5-> 정리할 수 없으면 $ k $개일 때 정리 불가.
        \end{itemize}
        \onslide<6-> 전반적인 구현 : priority queue 등의 자료구조를 이용
        
        \onslide<7-> 총 복잡도는 $ O(N\, \lg\, N \lg(A+B)) $.
    \end{frame}
    
    \begin{frame}[fragile]
        Code : \url{http://boj.kr/f3599f35f0f64a988dd3413d492f2f36}
        \begin{itemize}
            \item C++의 \verb|priority_queue|는 max-heap이라, 큰 값이 먼저 올라온다.
            \item \verb|priority_queue|에 원하는 비교 연산자를 넣는 방법?
            \begin{itemize}
            \item C++11의 경우 \verb|auto f = [] () {};| 꼴로 익명 함수를 통해 비교 함수를 만들고, \verb|priority_queue<T, vector<T>, decltype(f)> pq(f);|꼴로 선언하면 된다 (\verb|T|는 타입).
            \end{itemize}
        \end{itemize}
    \end{frame}

\end{document}

    %%%%%%%%%%%%%%%%%%%%%%%%%%%%%%%%%%%%%%%%%%%%%%%%%%%%%%%%%%%%%%%%%%%%%%%%%%%
    
    \section{Problem F}
    \subsection{A highway and the seven dwarfs (BOJ 7057)}
    \insertsectionpage{CEOI 2002 4번}{2}{1}{10}
    \documentclass[hyperref={unicode}]{beamer}

\usetheme{CambridgeUS}
\usecolortheme{beaver}
\usepackage[hangul]{kotex}

\usepackage{tabularx}
\usepackage{xcolor}
\usepackage{graphicx}
\usepackage[normalem]{ulem}
\usepackage{amsmath, amssymb}
\usepackage{colortbl}
\usepackage{textcomp}
\usepackage{multirow}
\setbeamertemplate{items}[circle]
\usepackage{minted}


\begin{document}
 
    \begin{frame}
        \onslide<1-> 직선이 평면에 있는 점들을 분할하는지를 확인하자.
        \begin{itemize}
            \item<2-> convex hull을 만들고, 직선이 이를 통과하는지 판별하는 것과 동치.
        \end{itemize}
        \onslide<3-> standard한 문제이기에 다양한 풀이가 있다.
        \begin{itemize}
            \item<4-> 여기서는 보다 간결한 이분탐색을 이용한 풀이를 설명한다.
        \end{itemize}
        
    \end{frame}
    
    \begin{frame}
        \onslide<1-> convex hull을 통과하는 방향을 생각해보자.
        \begin{itemize}
            \item<2-> 이 방향에서 가장 멀리 떨어진 두 점을 생각해볼 수 있다.
            \item<3-> 각 방향별로 이 두 점을 알면...?
            \begin{itemize}
                \item<4-> 두 점을 연결한 선분과 직선이 교차하는지 판별하면 된다!
            \end{itemize}
            \item<5-> 그럼 각도를 조금씩 기울여보자. 언제 두 점이 바뀔까?
            \item<6-> 직선의 기울기가 두 점이 속한 convex hull의 선분의 기울기를 넘어서면 된다!
        \end{itemize}
    \end{frame}
    
    \begin{frame}
        \begin{center}
            \begin{tikzpicture}[line cap=round,line join=round,>=triangle 45,x=1cm,y=1cm, scale=0.3]
            \clip(-18.896892689405195,-12.518551037588734) rectangle (0.6303288827343082,10.016027443340084);
            \draw [line width=1.5pt] (-11.22606159565516,4.5915000519107005)-- (-6.890421054925437,4.006919304846019);
            \draw [line width=1.5pt] (-6.890421054925437,4.006919304846019)-- (-5.0960082449606094,1.6531034367065363);
            \draw [line width=1.5pt] (-5.0960082449606094,1.6531034367065363)-- (-4.18,-2.5);
            \draw [line width=1.5pt] (-4.18,-2.5)-- (-6.372730499264858,-7.09482217668112);
            \draw [line width=1.5pt] (-6.372730499264858,-7.09482217668112)-- (-10.616362022738443,-8.852047409064667);
            \draw [line width=1.5pt] (-10.616362022738443,-8.852047409064667)-- (-12.8,-7.46);
            \draw [line width=1.5pt] (-12.8,-7.46)-- (-13.94,-3.56);
            \draw [line width=1.5pt] (-11.22606159565516,4.5915000519107005)-- (-13.94,-3.56);
            \draw [line width=1.5pt,color=ccwwff,domain=-18.896892689405195:0.6303288827343082] plot(\x,{(-2.2823173875755023--1.235346774086635*\x)/8.09154325096045});
            \begin{scriptsize}
            \draw [fill=ffffff] (-11.22606159565516,4.5915000519107005) circle (10pt);
            \draw [fill=wqwqwq] (-6.890421054925437,4.006919304846019) circle (10pt);
            \draw [fill=wqwqwq] (-5.0960082449606094,1.6531034367065363) circle (10pt);
            \draw [fill=wqwqwq] (-4.18,-2.5) circle (10pt);
            \draw [fill=wqwqwq] (-6.372730499264858,-7.09482217668112) circle (10pt);
            \draw [fill=ffffff] (-10.616362022738443,-8.852047409064667) circle (10pt);
            \draw [fill=wqwqwq] (-12.8,-7.46) circle (10pt);
            \draw [fill=wqwqwq] (-13.94,-3.56) circle (10pt);
            \end{scriptsize}
            \end{tikzpicture}
        \end{center}
    \end{frame}
    
    \begin{frame}
        \begin{center}
            \begin{tikzpicture}[line cap=round,line join=round,>=triangle 45,x=1cm,y=1cm, scale=0.3]
            \clip(-18.896892689405195,-12.51855103758873) rectangle (0.6303288827343082,10.016027443340082);
            \draw [line width=1.5pt] (-11.22606159565516,4.5915000519107005)-- (-6.890421054925437,4.006919304846019);
            \draw [line width=1.5pt] (-6.890421054925437,4.006919304846019)-- (-5.0960082449606094,1.6531034367065363);
            \draw [line width=1.5pt] (-5.0960082449606094,1.6531034367065363)-- (-4.18,-2.5);
            \draw [line width=1.5pt] (-4.18,-2.5)-- (-6.372730499264858,-7.09482217668112);
            \draw [line width=1.5pt] (-6.372730499264858,-7.09482217668112)-- (-10.616362022738443,-8.852047409064667);
            \draw [line width=1.5pt] (-10.616362022738443,-8.852047409064667)-- (-12.8,-7.46);
            \draw [line width=1.5pt] (-12.8,-7.46)-- (-13.94,-3.56);
            \draw [line width=1.5pt] (-11.22606159565516,4.5915000519107005)-- (-13.94,-3.56);
            \draw [line width=1.5pt,dotted,color=ccwwff,domain=-18.896892689405195:0.6303288827343082] plot(\x,{(-2.2701--1.24*\x)/8.09});
            \draw [line width=1.5pt,color=ccwwff,domain=-18.896892689405195:0.6303288827343082] plot(\x,{(--30.476411575757446--4.170149386117076*\x)/7.042411099720561});
            \begin{scriptsize}
            \draw [fill=ffffff] (-11.22606159565516,4.5915000519107005) circle (10pt);
            \draw [fill=wqwqwq] (-6.890421054925437,4.006919304846019) circle (10pt);
            \draw [fill=wqwqwq] (-5.0960082449606094,1.6531034367065363) circle (10pt);
            \draw [fill=wqwqwq] (-4.18,-2.5) circle (10pt);
            \draw [fill=ffffff] (-6.372730499264858,-7.09482217668112) circle (10pt);
            \draw [fill=wqwqwq] (-10.616362022738443,-8.852047409064667) circle (10pt);
            \draw [fill=wqwqwq] (-12.8,-7.46) circle (10pt);
            \draw [fill=wqwqwq] (-13.94,-3.56) circle (10pt);
            \end{scriptsize}
            \end{tikzpicture}
        \end{center}
    \end{frame}
    
    \begin{frame}
        \onslide<1-> convex hull을 위쪽과 아래쪽으로 쪼개자.
        \begin{itemize}
            \item<2-> Andrew's Monotone Chain Algorithm으로 쉽게 구축 가능.
        \end{itemize}
        \onslide<3-> 양 반껍질은 기울기 순으로 정렬이 되어 있다.
        \begin{itemize}
        \item<4-> 여기서 이분탐색을 진행해서 해당 기울기 이상이 되는 점을 잡자.
        \item<5-> 이 점이 주어진 직선의 기울기로부터 `가장 멀리 떨어져있는 점'이 된다! 
        \item<6-> 선분과 직선이 만나는지 확인만 하면 끝.
        \end{itemize}
    \end{frame}
    
    \begin{frame}
        Code : \url{http://boj.kr/514a4fadbe0d4521b22c45bef6bb8cea}
        \begin{itemize}
            \item {\color{blue}\href{https://en.wikibooks.org/wiki/Algorithm_Implementation/Geometry/Convex_hull/Monotone_chain}{Andrew's Monotone Chain Algorithm}}
            \item CCW도 익숙해집시다
        \end{itemize}
        
    \end{frame}
\end{document}
    
    %%%%%%%%%%%%%%%%%%%%%%%%%%%%%%%%%%%%%%%%%%%%%%%%%%%%%%%%%%%%%%%%%%%%%%%%%%%
    \section{분류}
    
    \begin{frame}
        \begin{enumerate}%[label=*\Alph]
            \item[A.] 낚시, 그리디
            \item[B.] 스위핑
            \item[C.] 트리
            \item[D.] 구현, 정수론
            \item[E.] 이분 매칭
            \item[F.] 기하, 볼록 껍질
            \item[G.] 동적 계획법
            \item[H.] 구현
            \item[I.] 이분탐색, 우선순위 큐, 그리디
        \end{enumerate}
    \end{frame}
    
\end{document}